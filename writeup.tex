% !TEX TS-program = pdflatex
% !TEX encoding = UTF-8 Unicode

\documentclass[12pt]{article}

%\usepackage{a4wide}
\usepackage[utf8]{inputenc}
\usepackage{amsmath}
\usepackage{amsfonts}
\usepackage{amssymb}
\usepackage{amsthm}
\usepackage{bm}
\usepackage{mathrsfs}
\usepackage{authblk}

% Bibliography
\usepackage[numbers]{natbib}

\usepackage{hyperref}
\hypersetup{
colorlinks=true,
linkcolor=blue,
filecolor=blue,
urlcolor=blue,
citecolor=blue
}

\newcommand{\dd}{\mathrm{d}}

%%% PAGE DIMENSIONS
\usepackage{geometry}
\geometry{
a4paper,
left=1.25in,
right=1.25in,
top=1.25in,
bottom=1.25in
}

\usepackage{graphicx}
\usepackage{float}
\usepackage{caption}

%%% PACKAGES
\usepackage{booktabs}
\usepackage{array}
\usepackage{paralist}
\usepackage{verbatim}
\usepackage{subfig}
\usepackage{enumitem}

%%% HEADERS & FOOTERS
\usepackage{fancyhdr}
\pagestyle{fancy}
\renewcommand{\headrulewidth}{0pt}
\lhead{}\chead{}\rhead{}
\lfoot{}\cfoot{\thepage}\rfoot{}

%%% SECTION TITLE APPEARANCE
\usepackage{sectsty}
\allsectionsfont{\sffamily\mdseries\upshape}

%%% ToC APPEARANCE
\usepackage[nottoc,notlof,notlot]{tocbibind}
\usepackage[titles,subfigure]{tocloft}
\renewcommand{\cftsecfont}{\rmfamily\mdseries\upshape}
\renewcommand{\cftsecpagefont}{\rmfamily\mdseries\upshape}

\usepackage[parfill]{parskip}

%%% THEOREM ENVIRONMENTS
\newtheorem{theorem}{Theorem}[section]
\newtheorem{lemma}[theorem]{Lemma}
\newtheorem{proposition}[theorem]{Proposition}
\newtheorem{corollary}[theorem]{Corollary}
\theoremstyle{definition}
\newtheorem{definition}[theorem]{Definition}
\theoremstyle{remark}
\newtheorem{remark}[theorem]{Remark}

%%% NOTATION SHORTCUTS
\newcommand{\R}{\mathbb{R}}
\newcommand{\N}{\mathbb{N}}
\newcommand{\Z}{\mathbb{Z}}
\newcommand{\Linf}{L^\infty}
\newcommand{\Lone}{L^1}
\newcommand{\supp}{\operatorname{supp}}

%%% TARGET BOUND — update when final result is obtained
\newcommand{\cTargetResult}{1.30}

%%% END preamble

\title{On Suprema of Autoconvolutions with an Application to Sidon Sets:\\
A GPU-Accelerated Approach}

\author[1]{Andrei Piterbarg}
\author[1]{Jai Bajaj}
\author[1]{Derrick Vincent}
\author[2]{Damek Davis}

\affil[1]{\it The University of Pennsylvania}
\affil[2]{\it The Wharton School, University of Pennsylvania}

\renewcommand\Authands{ and }

\begin{document}

\maketitle
\vskip 0.5in

%% ================================================================
%%  ABSTRACT
%% ================================================================

\begin{abstract}
Let $f$ be a nonnegative function supported on $(-1/4, 1/4)$. We study the
autoconvolution constant $c$ defined by
\[
  \max_{-1/2 \leq t \leq 1/2} \int_\R f(t - x) f(x) \, dx
  \;\geq\; c \left( \int_{-1/4}^{1/4} f(x) \, dx \right)^{2}
\]
which arises naturally in the study of Sidon sets in additive combinatorics.
The previous best lower bound $c \geq 1.28$ was established by Cloninger and
Steinerberger~\cite{CS14} using a branch-and-prune algorithm that required
approximately 20{,}000 CPU hours.
We present a GPU-accelerated reimplementation targeting
NVIDIA A100 GPUs that achieves throughputs exceeding
$10^9$ configurations per second---roughly three orders of magnitude faster
than the original MATLAB implementation---pushing
the proven lower bound to $c \geq \cTargetResult$.
\end{abstract}


%% ================================================================
\section{Introduction}\label{sec:intro}
%% ================================================================

% ------------------------------------------------------------------
\subsection{Sidon sets and the autoconvolution constant}\label{ssec:sidon}
% ------------------------------------------------------------------

A subset $A \subset \{1, 2, \ldots, N\}$ is called \emph{$g$-Sidon} if every
integer has at most $g$ representations as a sum $a + b$ with $a, b \in A$.
The maximum size $\beta_g(N)$ of such a set satisfies, as shown by
Cilleruelo, Ruzsa, and Vinuesa~\cite{CRV},
\[
  \beta_g(N) \;\sim\; \sigma\,\sqrt{gN}
  \qquad (g \to \infty,\; N \to \infty),
\]
where $\sigma$ is a universal constant whose precise value remains open.

The constant $\sigma$ admits an equivalent continuous characterization~\cite{CRV, SS}.
Given a nonnegative function $f$ supported on $(-1/4, 1/4)$, define its
\emph{autoconvolution} $(f * f)(x) = \int_{\R} f(t)\,f(x - t)\,dt$.
Setting $c = 2/\sigma^2$, one has
\[
  \|f * f\|_{\Linf(\R)}
  \;\geq\; c \left(\int_{-1/4}^{1/4} f(x)\,dx\right)^{\!2}
\]
for every such~$f$. A larger $c$ means that no matter how one
distributes nonnegative mass on $(-1/4, 1/4)$, the supremum of $f * f$ cannot
be made small.

% ------------------------------------------------------------------
\subsection{Prior work}\label{ssec:prior}
% ------------------------------------------------------------------

A succession of authors have improved the lower bound on $c$:

\medskip
\begin{center}
\begin{tabular}{@{}ll@{}}
\toprule
\textbf{Lower bound} & \textbf{Source} \\
\midrule
$c \geq 1$       & trivial \\
$c \geq 1.151$   & Cilleruelo, Ruzsa, and Trujillo~\cite{CRT} \\
$c \geq 1.178$   & Green~\cite{Green} \\
$c \geq 1.183$   & Martin and O'Bryant~\cite{MO1} \\
$c \geq 1.251$   & Yu~\cite{Yu} \\
$c \geq 1.263$   & Martin and O'Bryant~\cite{MO2} \\
$c \geq 1.274$   & Matolcsi and Vinuesa~\cite{MV} \\
$c \geq 1.28$    & Cloninger and Steinerberger~\cite{CS14} \\
$c \geq \cTargetResult$    & this work \\
\bottomrule
\end{tabular}
\end{center}
\medskip

\noindent
On the upper bound side, Matolcsi and Vinuesa~\cite{MV} proved $c \leq 1.5099$
via an explicit construction, disproving the conjecture $c = \pi/2$ of Schinzel
and Schmidt~\cite{SS}. This was recently sharpened to $c \leq 1.5029$ by
TTT-Discover~\cite{TTT}. The state of the art prior to this work was therefore
$1.28 \leq c \leq 1.5029$.

The bound $c \geq 1.28$ of Cloninger and Steinerberger~\cite{CS14}
introduced a fundamentally different approach: rather than Fourier
analysis (whose method has a theoretical ceiling of~$1.276$~\cite{MV}), they designed a branch-and-prune algorithm
that reduces the problem to a \emph{finite} verification, requiring
approximately 20{,}000 CPU hours.

% ------------------------------------------------------------------
\subsection{Our contribution}\label{ssec:contribution}
% ------------------------------------------------------------------

We present a GPU-native reimplementation of this algorithm
targeting NVIDIA A100 GPUs. The mathematical framework is unchanged;
our contribution is computational. The key ideas are:
(i)~fused single-pass CUDA kernels that generate, prune, and evaluate
each composition entirely in registers, eliminating all intermediate
global memory traffic;
(ii)~a multi-stage FP32 pruning cascade that eliminates ${\sim}99\%$ of
configurations before the expensive FP64 autoconvolution;
and (iii)~incremental convolution that halves the inner-loop cost for
$d \geq 6$.
These yield throughputs exceeding $10^9$ configurations/second,
enabling exhaustive verification at previously infeasible parameter
regimes and proving $c \geq \cTargetResult$.


%% ================================================================
\section{The Branch-and-Prune Algorithm}\label{sec:algorithm}
%% ================================================================

We recall the algorithm of Cloninger and Steinerberger~\cite{CS14}.

% ------------------------------------------------------------------
\subsection{Reduction to bin averages}\label{ssec:bin-averages}
% ------------------------------------------------------------------

\begin{lemma}[Bin averages, {\cite[Lemma~1]{CS14}}]
\label{lem:step}
For any $n \in \N$, define the simplex
\[
  A_n = \bigl\{(a_{-n},\, a_{-n+1},\, \ldots,\, a_{n-1}) \in \R_+^{2n}
    : {\textstyle\sum_{i=-n}^{n-1}} a_i = 4n \bigr\}
\]
and the minimax value
\[
  a_n := \min_{a \in A_n}\;
  \max_{\substack{2 \leq \ell \leq 2n \\ -n \leq k \leq n - \ell}}
  \frac{1}{4n\ell} \sum_{\substack{i,j \\ k \leq i+j \leq k+\ell-2}} a_i\, a_j,
\]
where $k, \ell \in \Z$. Then $c \geq a_n$.
\end{lemma}

\begin{proof}
Let $\varepsilon > 0$. Choose $f \geq 0$ supported on $(-1/4,\, 1/4)$ with $\int f = 1$ and $\|f * f\|_{\Linf} \leq c + \varepsilon$.
Partition $(-1/4,\, 1/4)$ into $2n$ equal sub-intervals
$I_j = \bigl(\tfrac{j}{4n},\; \tfrac{j+1}{4n}\bigr)$
and define $a_j = 4n \int_{I_j} f$.
Then $a \in A_n$, and for any window $(k, \ell)$:
\begin{align*}
  \frac{1}{4n\ell} \sum_{k \leq i+j \leq k+\ell-2} a_i\, a_j
  &= \frac{4n}{\ell} \sum_{k \leq i+j \leq k+\ell-2}
     \int_\R (f_i * f_j)(x)\,dx \\
  &\leq \frac{4n}{\ell} \int_{k/(4n)}^{(k+\ell)/(4n)} (f * f)(x)\,dx
  \leq \|f * f\|_{\Linf}
  \leq c + \varepsilon,
\end{align*}
where the first inequality uses $\supp(f_i * f_j) \subseteq I_i + I_j \subseteq \bigl(\tfrac{k}{4n},\, \tfrac{k+\ell}{4n}\bigr)$.
Since $\varepsilon$ was arbitrary, $c \geq a_n$.
\end{proof}

% ------------------------------------------------------------------
\subsection{Discretization}\label{ssec:discretization}
% ------------------------------------------------------------------

\begin{lemma}[Discretization bound, {\cite[Lemmas~2--3]{CS14}}]
\label{lem:disc}
For any $n, m \in \N$, define the discrete lattice
\[
  B_{n,m} = \left\{b \in \left(\tfrac{1}{m}\N\right)^{2n}
    : {\textstyle\sum} b_i = 4n\right\}.
\]
Then $B_{n,m}$ is a $1/m$-net of $A_n$ in the $\ell^\infty$-norm.
Defining
\[
  b_{n,m} := \min_{b \in B_{n,m}}\;
  \max_{\substack{2 \leq \ell \leq 2n \\ -n \leq k \leq n - \ell}}
  \frac{1}{4n\ell} \sum_{k \leq i+j \leq k+\ell-2} b_i\, b_j,
\]
we have $c \geq b_{n,m} - 2/m - 1/m^2$.
\end{lemma}

\begin{proof}
For any $a \in A_n$, a greedy rounding procedure produces $b \in B_{n,m}$ with $\|a - b\|_\infty \leq 1/m$.
Writing $a = b + \varepsilon$ with $\|\varepsilon\|_\infty \leq 1/m$, the cross-terms satisfy
$|(f * \varepsilon)(x)| \leq 1/m$ and $|(\varepsilon * \varepsilon)(x)| \leq 1/m^2$.
Combining with Lemma~\ref{lem:step} gives the result.
\end{proof}

\begin{remark}[Refined correction]\label{rem:refined}
In practice, a tighter bound $|(f * \varepsilon)(x)| \leq \frac{1}{m}\int_{-1/4 + \max(x,0)}^{1/4 + \min(0,x)} f(y)\,dy$ yields window-dependent corrections that are smaller near the boundary of $(-1/2,\, 1/2)$, improving pruning efficiency.
\end{remark}

% ------------------------------------------------------------------
\subsection{Verification and hierarchical refinement}\label{ssec:verification}
% ------------------------------------------------------------------

To prove $c \geq c_{\mathrm{target}}$, define the effective threshold
$T = c_{\mathrm{target}} + 2/m + 1/m^2$
and show that \emph{every} $b \in B_{n,m}$ has at least one window $(k,\ell)$ whose average exceeds~$T$.
Such a configuration is \emph{ruled out}; if all are ruled out, the bound is proven.

The lattice size $|B_{n,m}| = \binom{4nm + 2n - 1}{2n - 1}$ grows exponentially in~$n$, making direct enumeration at large~$n$ infeasible.
However, many configurations can be ruled out at coarse resolution, motivating a hierarchical strategy:

\begin{enumerate}[label=\textbf{Step \arabic*.},leftmargin=*]
\item \textbf{Base level.}
Enumerate all $b \in B_{n_0,m}$ at a coarse discretization (e.g., $n_0 = 3$, $d = 6$ bins) and test each against~$T$.
Survivors are those not yet ruled out.

\item \textbf{Refinement.}
Each surviving parent $b$ is refined by splitting each bin $b_i$ into two sub-components $c_{2i} + c_{2i+1} = b_i$, producing children at resolution $d' = 2d$.
The number of children per parent is $N_b = \prod_{i}(1 + m\,b_i)$.
A parent is ruled out iff \emph{every} child is ruled out.

\item \textbf{Iterate.}
Repeat along the dyadic schedule $d = 6, 12, 24, 48, \ldots$ until all configurations are eliminated.
\end{enumerate}

Two simple pruning rules reduce the search space.
First, the palindromic symmetry $b \leftrightarrow \mathrm{rev}(b)$ allows restricting to canonical compositions with $b \leq \mathrm{rev}(b)$ lexicographically, halving the work.
Second, if the mass fraction on one half of the support satisfies $\sum_{i < 0} a_i / \sum a_i \geq \sqrt{c_{\mathrm{target}}/2}$, the configuration is immediately ruled out by an asymmetry bound without computing the full autoconvolution.


%% ================================================================
\section{GPU Implementation}\label{sec:implementation}
%% ================================================================

We describe our CUDA implementation targeting NVIDIA A100 GPUs (80\,GB HBM2e, 108 streaming multiprocessors).

% ------------------------------------------------------------------
\subsection{Fused single-pass kernels}\label{subsec:fused}
% ------------------------------------------------------------------

A naive GPU pipeline would separate composition generation, pruning, and test-value evaluation into three stages with intermediate results stored in global DRAM.
For large $m$, the number of compositions can exceed $10^9$, making intermediate storage prohibitive.

Our fused kernel performs the entire pipeline within a single kernel invocation.
Each thread is assigned a flat work index, which it decodes into a composition $\mathbf{c} = (c_0, \ldots, c_{d-1})$ via closed-form simplex inversion (no shared-memory coordination).
The thread then applies pruning tests, computes the autoconvolution, and scans all windows---entirely within registers.
The only global memory writes are per-block reduction outputs (at most 3{,}456 values for the full A100), which are reduced on the host after kernel completion.

To balance workload, the first component $c_0$ is processed in zigzag order (alternating small and large values), so that each thread's work chunk contains a mix of combinatorially expensive and cheap slices.
Per-$c_0$ composition counts are prefix-summed on the host, enabling $O(\log n)$ binary search from flat index to $c_0$ on the device.

% ------------------------------------------------------------------
\subsection{Multi-stage pruning cascade}\label{subsec:pruning}
% ------------------------------------------------------------------

We apply a hierarchy of increasingly expensive pruning tests, eliminating ${\sim}99\%$ of compositions before the costly FP64 autoconvolution:

\begin{enumerate}
\item \textbf{Canonical filter.} Skip compositions violating $\mathbf{c} \leq \mathrm{rev}(\mathbf{c})$.

\item \textbf{FP32 asymmetry bound.}
If the mass fraction $p$ on one side satisfies $2(\max(p, 1{-}p) - \delta)^2 > T$ (where $\delta = 1/(4m)$), prune immediately.

\item \textbf{FP32 block-sum bounds.}
For selected windows, check whether the squared partial sum already exceeds the threshold---e.g., $(c_0 + \cdots + c_{d/2-1})^2$ at $\ell = d$, or adjacent pairs $(c_i + c_{i+1})^2$ at $\ell = 4$.

\item \textbf{FP32 max-element bound.}
At $\ell = 2$, the test value is at least $(\max_i c_i)^2 / (8n) \cdot (4n/m)^2$.

\item \textbf{FP64 asymmetry recheck.} Re-evaluate the asymmetry bound in double precision.

\item \textbf{Central convolution bound} ($d = 6$). If the single entry $\mathrm{conv}[d{-}1] = 2(c_0 c_5 + c_1 c_4 + c_2 c_3)$ already exceeds the $\ell = 2$ threshold, prune.
\end{enumerate}

All FP32 stages use a safety margin of $10^{-5}$ to prevent incorrect pruning due to rounding.

% ------------------------------------------------------------------
\subsection{Autoconvolution and incremental computation}\label{subsec:autoconv}
% ------------------------------------------------------------------

For surviving compositions, we compute the discrete autoconvolution in integer arithmetic (INT32 when $S^2 \leq 2 \times 10^9$, INT64 otherwise):
\[
    \mathrm{conv}[k] = \sum_{\substack{0 \leq i, j < d \\ i + j = k}} c_i \cdot c_j, \qquad k = 0, \ldots, 2d-2.
\]
The array is prefix-summed in-place, enabling $O(1)$ extraction of any window sum.
The test value at window $(\ell, s)$ is
\[
    \mathrm{tv}(\ell, s) = \frac{4n}{m^2 \ell} \bigl(\mathrm{conv}[s+\ell-2] - \mathrm{conv}[s-1]\bigr),
\]
with early exit once any window exceeds the threshold.

\paragraph{Incremental convolution for $d = 6$.}
The components $(c_0, c_1, c_2, c_3)$ are fixed by the flat-index decoding, while $c_4$ varies in an inner loop with $c_5 = S - c_0 - \cdots - c_4$.
The prefix-summed entries $\mathrm{conv}[0], \ldots, \mathrm{conv}[3]$ depend only on $(c_0, \ldots, c_3)$ and are computed once outside the loop.
For $k \geq 4$, each entry decomposes as
$\mathrm{conv}[k] = \mathrm{conv}[k{-}1] + B_k + \text{(cross-terms involving $c_4, c_5$)}$,
where the base terms $B_k$ are precomputed.
This reduces the inner-loop cost from 21 to 11 multiplications per iteration.

\paragraph{Batched refinement.}
At refinement levels, multiple parents are packed into a single kernel launch via prefix-sum indexing of child counts, eliminating per-parent kernel launch overhead.
An energy cap $x_{\max} = \lfloor m\sqrt{T/d_{\mathrm{child}}} \rfloor$ avoids generating children that would be trivially pruned, often reducing child counts by an order of magnitude.
When the survivor count exceeds GPU memory, a chunked streaming pipeline writes survivors to disk with adaptive chunk sizing.


%% ================================================================
\section{Computational Results}\label{sec:results}
%% ================================================================

Using the implementation above on an NVIDIA A100-SXM4-80GB, we execute the branch-and-prune verification with $m = 50$ and $n_0 = 3$.
The level schedule is $d \in \{6, 12, 24, 48\}$ bins.
The effective threshold is $T = c_{\mathrm{target}} + 2/50 + 1/2500 = c_{\mathrm{target}} + 0.0404$.

The base-level enumeration ($d = 6$, $|B_{3,50}| = 3{,}478{,}761$ lattice points) completes in under one second, compared to hours in the original implementation.
The majority of runtime is spent at levels $d = 24$ and $d = 48$ where surviving parents generate large numbers of children.

% TODO: Add final runtime table and proven bound once computation completes.


%% ================================================================
\section{Conclusion}\label{sec:conclusion}
%% ================================================================

We have presented a GPU-accelerated reimplementation of the Cloninger--Steinerberger branch-and-prune algorithm, achieving throughputs exceeding $10^9$ configurations per second on an NVIDIA A100---roughly three orders of magnitude faster than the original MATLAB implementation.
This speedup enables exhaustive verification at previously infeasible parameter regimes, improving the lower bound to $c \geq \cTargetResult$.

Natural extensions include: higher base dimensions ($d = 8, 10, \ldots$) to reduce refinement levels; larger discretization parameters $m > 50$ to tighten the correction term; and multi-GPU scaling for the embarrassingly parallel enumeration and refinement stages.


%% ================================================================
%%  BIBLIOGRAPHY
%% ================================================================

\begin{thebibliography}{99}

\bibitem{CS14}
A.~Cloninger and S.~Steinerberger,
\emph{On suprema of autoconvolutions with an application to Sidon sets},
Proc.\ Amer.\ Math.\ Soc.\ \textbf{145} (2017), no.~8, 3191--3200.
\href{https://arxiv.org/abs/1403.7988}{arXiv:1403.7988}.

\bibitem{CRV}
J.~Cilleruelo, I.~Ruzsa, and C.~Vinuesa,
\emph{Generalized Sidon sets},
Adv.\ Math.\ \textbf{225} (2010), no.~5, 2786--2807.

\bibitem{CRT}
J.~Cilleruelo, I.~Ruzsa, and C.~Trujillo,
\emph{Upper and lower bounds for finite $B_h[g]$ sequences},
J.\ Number Theory \textbf{97} (2002), no.~1, 26--34.

\bibitem{Green}
B.~Green,
\emph{The number of squares and $B_h[g]$ sets},
Acta Arith.\ \textbf{100} (2001), 365--390.

\bibitem{MO1}
G.~Martin and K.~O'Bryant,
\emph{The supremum of autoconvolutions, with applications to additive number theory},
Illinois J.\ Math.\ \textbf{53} (2009), no.~1, 219--235.

\bibitem{MO2}
G.~Martin and K.~O'Bryant,
\emph{Constructions of generalized Sidon sets},
J.\ Combin.\ Theory Ser.~A \textbf{113} (2006), no.~4, 591--607.

\bibitem{MV}
M.~Matolcsi and C.~Vinuesa,
\emph{Improved bounds on the supremum of autoconvolutions},
J.\ Math.\ Anal.\ Appl.\ \textbf{372} (2010), no.~2, 439--447.

\bibitem{SS}
A.~Schinzel and W.~M.~Schmidt,
\emph{Comparison of $L^1$- and $L^\infty$-norms of squares of polynomials},
Acta Arith.\ \textbf{104} (2002), 283--296.

\bibitem{Yu}
G.~Yu,
\emph{An upper bound for $B_2[g]$ sets},
J.\ Number Theory \textbf{122} (2007), 211--220.

\bibitem{TTT}
TTT-Discover,
\emph{Improved upper bound on the supremum of autoconvolutions},
2025.

\end{thebibliography}

\end{document}
